\section{Conclusion}\label{sec: Conclusion}
In this thesis, an approach of a knowledge-based design of a customizable product configurator is given. The configurator was demonstrated by using a bike configuration along with three different use cases of user requirements. Also, the configuration knowledge can be applied to different product domains. \newline

The scientific foundation was build by conducting a systematic literature review in which the different key elements of a configuration were analysed. Therefore, I first looked at the terminology and ontology for product configuration. Afterward, I defined the process of knowledge acquisition and analysed different approaches to configuration knowledge representation. After identifying ASP as a suitable representation for product configuration the semantics and syntax of ASP were defined. At the end of the literature review, I covered interactive configuration (including configuration diagnosis and explanation) and reconfiguration. \newline

Before a problem can be solved, it needs to be defined. Therefore, I briefly described the bike using a UML class diagram and a supporting table (appendix). In addition, I set up rules for a general configuration problem where I transformed the results of my literature review into requirements for a configurator. Afterward, I defined three different use cases with different user requirements. The first user required a city bike, the second a mountain bike, and the third was very price sensitive and always chose the rather inexpensive option. \newline

To solve the problem I first translated the knowledge of the product domain into facts. Then, I introduced a domain-independent ASP encoding of the configuration knowledge. To show how the configurator works, I applied the different user requirements to the configurator, respectively.  \newline

So far, research about product configuration in ASP has mainly focused on creating product configurators for specific product domains or the domain knowledge is expressed in rule-format. Some authors even state that it is not possible to create a domain-independent product configurator \cite{junker06a}. When using rules to represent the product domain, domain experts are required to also gain expertise in ASP. However, this cannot be implied for a company where the domain experts (for example product managers) have little to no IT background \cite{zhang14a}. By separating the domain knowledge and the configuration knowledge and representing the domain knowledge in fact format I overcame both problems. The configurator can be applied to different product domains and no in-depth knowledge about ASP for the domain expert is required. 
\newpage
\subsection{Future Work}
The presented concept and encoding can be further improved in different areas. Looking at use case 3, the user required always the cheapest component type which was no problem for a relatively simple product like a bike. For more complex configuration problems it is a good idea to include optimization criteria. ASP offers the \textit{minimize} and \textit{maximize} functions which could be applied to the price or to other resource constraints \cite{gekakasc11d}. \newline

Furthermore, not all constraints were included in the ASP encoding. For future work, it is important to include the resource rules, if applicable the compatibility instead of incompatibility constraints, and the default values. The default values can also be used to introduce an interactive product configuration. For the user interaction, it is important to include component suggestions (for example in the form of default values) and rank user preferences. In addition, a diagnosis needs to be implemented in the case that a configuration is not satisfiable \cite{syrjanen00a}. \newline

For a successful configuration process, customer interaction is crucial. Therefore, options for interactive configuration can be added in the future. Customer interaction requires both a frontend (user interface) and a backend (KRR system). For the backend, a recommender system 
to deliver an optimal solution, provide a diagnosis, and communicate with the user \cite{fahakrscscta20a} can be added. 

